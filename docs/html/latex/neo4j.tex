Neo4j is a graph based database which stores data in Nodes \& Edges. We start by retrieving our component factory \& interface. \begin{DoxyVerb}#include "include/aossl/neo4j/neo4j_interface.h"
#include "include/aossl/neo4j/factory_neo4j.h"

Neo4jComponentFactory neo4j_factory;
Neo4jInterface *neo = neo4j_factory.get_neo4j_interface("neo4j://localhost:7687");
\end{DoxyVerb}


Now, we can execute Cypher queries with the command below\+: \begin{DoxyVerb}std::string query = "CREATE (you:Person {name:'E', list: [1, 2, 3]})-[like:LIKE]->(me:Person {name:'F', list: [4, 5, 6]}) RETURN you, like, me";
ResultsIteratorInterface *results = neo->execute(query);
\end{DoxyVerb}


Our results iterator lets us view the results of the query\+: \begin{DoxyVerb}ResultTreeInterface* result = results->next();
DbObjectInterface* obj = result->get(0);
std::cout << obj->to_string() << std::endl;
\end{DoxyVerb}


The result tree represents a single set of results. This can include many nodes and/or edges, and depends entirely on the R\+E\+T\+U\+RN clause of the cypher query. We can retrieve values by index from this tree into the Db\+Object, which is either a node, edge, or path.

We can access properties of nodes \& edges\+: \begin{DoxyVerb}DbMapInterface* map = obj->properties();
if (map->element_exists("name")) {
  std::cout << map->get_string_element("name") << std::endl;
}
if (map->element_exists("list")) {
  DbListInterface *list_prop = map->get_list_element("list");
  std::cout << "List Size: " << list_prop->size() << std::endl;
  std::cout << list_prop->get_int_element(0) << std::endl;
  delete list_prop;
}
\end{DoxyVerb}


As well as labels of nodes and directions of edges\+: \begin{DoxyVerb}DbListInterface* list = obj->labels();
std::cout << list->get_string_element(0) << std::endl;
DbObjectInterface* edge = result->get(1);
if (edge->forward()) {
  std::cout << "Forward edge" << std::endl;
}
\end{DoxyVerb}


Remember to delete the objects that you\textquotesingle{}ve used when viewing the results! \begin{DoxyVerb}delete results;
delete result;
delete obj;
delete map;
delete list;
delete edge;
\end{DoxyVerb}


\subsection*{Query Parameters}

We can pass parameters to queries by creating named maps with any number of parameters\+: \begin{DoxyVerb}std::unordered_map<std::string, Neo4jQueryParameterInterface*> query_params;
Neo4jQueryParameterInterface* name_param = neo4j_factory.get_neo4j_query_parameter("E");
query_params.emplace("inp_name", name_param);
\end{DoxyVerb}


Then, we can use the name we define as a variable, surrounded by \char`\"{}\{\}\char`\"{} within the query, and we pass the map to the execute function. \begin{DoxyVerb}std::string query = "MATCH (you:Person) WHERE you.name = {inp_name} RETURN you"
ResultsIteratorInterface *results = neo->execute(query, query_params);
\end{DoxyVerb}


We can then access the results as above through the iterator.

\subsection*{Paths}

Certain queries return path objects instead of nodes and edges\+: \begin{DoxyVerb}std::string query = "MATCH (base:Person {name:'E'}), (next:Person {name:'F'}), p = shortestPath((base)-[r*]-(next)) RETURN p"
ResultsIteratorInterface *results = neo->execute(query);
\end{DoxyVerb}


Then, we can access the results\+: \begin{DoxyVerb}ResultTreeInterface* result = results->next();
DbObjectInterface* obj = result->get(0);
DbObjectInterface* path_obj = NULL;
if (obj->is_path()) {
  int path_size = path->size();
  DbObjectInterface* path_obj = path->get_path_element(i);
  std::cout << path_obj->to_string() << std::endl;
}
\end{DoxyVerb}


\hyperlink{index}{Go Home} 
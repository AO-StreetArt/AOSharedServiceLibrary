The Command\+Line\+Interpreter makes working with input parameters easier. With it, we get access to command line arguments that are entered in the form\+:

{\ttfamily ./example name=abc}

\subsection*{Use}

We start by importing the necessary interfaces and establish the service factory. \begin{DoxyVerb}#include "include/commandline/factory_cli.h"
#include "include/commandline/commandline_interface.h"

int main( int argc, char** argv )
{
CommandLineInterpreterFactory cli_factory;
CommandLineInterface *cli = cli_factory.get_command_line_interface( argc, argv );
\end{DoxyVerb}


We can utilize get\+\_\+program\+\_\+name to return the current program name executing on this instance. \begin{DoxyVerb}std::cout << cli->get_program_name() << std::endl;
\end{DoxyVerb}


We have access to an opt\+\_\+exist method to determine if an option was entered, and we can use get\+\_\+opt to pull parameter values. \begin{DoxyVerb}if ( cli->opt_exist("name") ) {
  std::cout << cli->get_opt("name") << std::endl;
}

return 0;
}
\end{DoxyVerb}


\subsection*{Multi-\/\+Threading}

The Command Line Interpreter is threadsafe, and it is recommended that you utilize a single interpreter and have each thread access the arguments in it, to ensure that the input parameters are only parsed once.

\hyperlink{index}{Go Home} 
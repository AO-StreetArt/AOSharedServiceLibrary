\section*{Consul Administrator}

Consul is responsible for Service Registration \& Discovery, as well as Key-\/\+Value Retrieval.

\subsection*{Use}

We start by importing the necessary interfaces and establish the service factory. \begin{DoxyVerb}#include "aossl/consul/consul_interface.h"
#include "aossl/consul/factory_consul.h"

ConsulComponentFactory consul_factory;
\end{DoxyVerb}


Next, we define a Service which represents the current instance of the code.

This contains an ID, a name, a connection, and a port number. Tags can be added or not\+: \begin{DoxyVerb}ServiceInterface *s = consul_factory.get_service_interface("1", "test", "http://localhost/", "5555");
s->add_tag("Testing");
\end{DoxyVerb}


And build our Consul Admin, specifying the connection address\+: \begin{DoxyVerb}ConsulInterface *ca = consul_factory.get_consul_interface("localhost:8500");
\end{DoxyVerb}


Now, we can register the currently running instance of a service with Consul \begin{DoxyVerb}ca->register_service(*s);
\end{DoxyVerb}


And, we can unregister on shutdown\+: \begin{DoxyVerb}ca->deregister_service(*s);
\end{DoxyVerb}


We can get and set key-\/value storage elements\+: \begin{DoxyVerb}bool success = ca->set_config_value("Test", "123");
assert(success);

std::string test_val = ca->get_config_value("Test");
logging->debug(test_val);
\end{DoxyVerb}


The Consul Admin also supports querying available services, nodes, and datacenters. Please see the A\+PI documentation for all available queries, as well as the \href{https://www.consul.io/api/index.html}{\tt Consul H\+T\+TP A\+PI} for details on expected responses.

All responses from the Consul Admin are strings which hold the J\+S\+ON response from the Consul H\+T\+TP A\+PI. Configuration Values retrieved from the KV store in Consul are also returned as J\+S\+ON, and the value returned is encoded in base64. A convenience method is provided for decoding the text from base64. Please see the Consul documentation for further details.

\subsection*{Multi-\/\+Threading}

The Consul Admin utilizes H\+T\+TP to communicate. It is threadsafe except for the methods marked as not, but has a mutex on the H\+T\+TP Send function, so only one value can be retrieved at a time.

\hyperlink{index}{Go Home} 
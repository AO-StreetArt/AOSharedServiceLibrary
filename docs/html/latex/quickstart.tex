\subsection*{Docker}

The AO Shared Service Library is available as a docker image. This allows developers to get a fully functional build environment with a single command, and offers a head-\/start on making application docker files.

\subsubsection*{Setup}

Before we begin, we need to have a few things ready.

\paragraph*{Installing Docker}

If you do not already have Docker installed, please follow the instructions \href{https://docs.docker.com/engine/installation}{\tt here}.

\subsubsection*{Running the Docker Image}

You can create a fully functional build environment for new micro services via Docker, and ssh into that process. \begin{DoxyVerb}docker run --name aossl -d -P aostreetart/ao-services
\end{DoxyVerb}


Then, you can access the container with the following\+: \begin{DoxyVerb}docker exec -i -t aossl /bin/bash
\end{DoxyVerb}


\subsubsection*{Docker Images of External Tools}

Docker images are also available for many of the external tools connected to within the library.

\paragraph*{Mongo}

In times when you need to connect to an instance of \href{https://www.mongodb.com}{\tt Mongo}, you can use the docker image (full instructions can be found \href{https://hub.docker.com/_/mongo/}{\tt here}). \begin{DoxyVerb}docker run --name some-mongo -d mongo
\end{DoxyVerb}


\paragraph*{Neo4j}

In times when you need to connect to an instance of \href{https://neo4j.com/}{\tt Neo4j}, you can use the docker image (full instructions can be found \href{https://hub.docker.com/_/neo4j/}{\tt here}). \begin{DoxyVerb}docker run \
    --publish=7474:7474 --publish=7687:7687 \
    --volume=$HOME/neo4j/data:/data \
    neo4j
\end{DoxyVerb}


\paragraph*{Redis}

In times when you need to connect to an instance of \href{http://redis.io/}{\tt Redis}, you can use the docker image (full instructions can be found \href{https://hub.docker.com/_/redis/}{\tt here}). \begin{DoxyVerb}docker run --name some-redis -d redis
\end{DoxyVerb}


\paragraph*{Consul}

In times when you need to connect to an instance of \href{https://www.consul.io/}{\tt Consul}, you can use the docker image (full instructions can be found \href{https://hub.docker.com/_/consul/}{\tt here}) \begin{DoxyVerb}docker run -d --name=dev-consul consul
\end{DoxyVerb}


\subsubsection*{Connecting Docker Images}

Connecting Docker images is done via the network command. First, we start the network\+: \begin{DoxyVerb}docker network create my-network
\end{DoxyVerb}


Then, we utilize the \&ndash;network option when starting containers to connect them to the network\+: \begin{DoxyVerb}docker run -d --name=registry --network=my-network consul
\end{DoxyVerb}


When we start another docker container and connect it to my-\/network, we can access the first container by using it\textquotesingle{}s container name as the hostname. For example, we\textquotesingle{}d access the consul agent started above from another docker container in the network at the address \textquotesingle{}registry\+:8500\textquotesingle{}.

\subsection*{Use Latest Release}

Please see the \href{https://github.com/AO-StreetArt/AOSharedServiceLibrary/releases}{\tt releases} page to download the latest release of the library. Once downloaded, unpack the tar/zip file and cd into the main directory. Then, run the following command\+: \begin{DoxyVerb}sudo ./easy_install
\end{DoxyVerb}


You will be prompted for your sudo password, after which the script will attempt to install all of the necessary dependencies, and then the library itself. If you prefer, you can simply run\+: \begin{DoxyVerb}sudo make install
\end{DoxyVerb}


This will install the library without installing the dependencies. You may execute the install dependencies script separately if desired via\+: \begin{DoxyVerb}cd deps && sudo ./build_deps.sh
\end{DoxyVerb}


Or, you may refer to the \href{https://github.com/AO-StreetArt/AOSharedServiceLibrary/tree/master/docs/deps}{\tt Dependency Resolution} section of the documentation on how to install necessary dependencies manually.

You may uninstall the library by executing\+: \begin{DoxyVerb}sudo make uninstall
\end{DoxyVerb}


\subsection*{Install the latest development versions}

Alternatively, you may clone the source from git directly and build the library yourself. Note that this is currently only recommended on Unix systems due to O\+S-\/level dependencies. Windows users should work with the Dockerfile provided.

\subsubsection*{Setup}

Before we begin, we need to build our dependencies and then build the project.

\paragraph*{Dependencies}

\subparagraph*{Ubuntu16.\+04/\+Debian 7}

The build\+\_\+deps.\+sh script should allow for automatic resolution of dependencies. Run the following commands from within the main folder \begin{DoxyVerb}mkdir ../aossl_deps

sudo cp scripts/deb/build_deps.sh ../aossl_deps

cd ../aossl_deps

sudo ./build_deps.sh
\end{DoxyVerb}


\subparagraph*{Cent\+OS 7/\+Redhat Enterprise Linux 7}

The build\+\_\+deps.\+sh script should allow for automatic resolution of dependencies. Run the following commands from within the main folder \begin{DoxyVerb}mkdir ../aossl_deps

sudo cp scripts/rhel/build_deps.sh ../aossl_deps

cd ../aossl_deps

sudo ./build_deps.sh
\end{DoxyVerb}


\subparagraph*{Other}

Please refer to the \href{https://github.com/AO-StreetArt/AOSharedServiceLibrary/tree/master/docs/deps}{\tt Dependency Resolution} section of the documentation.

\paragraph*{Build the Project}

The project and tests can be built with make on most linux systems. \begin{DoxyVerb}make
\end{DoxyVerb}


We can clean the build and remove all generated files with\+: \begin{DoxyVerb}make clean
\end{DoxyVerb}


\paragraph*{Install and Uninstall the Project}

The project can be installed on linux systems with\+: \begin{DoxyVerb}sudo make install
\end{DoxyVerb}


We can uninstall the libraries with\+: \begin{DoxyVerb}sudo make uninstall
\end{DoxyVerb}


\paragraph*{Build the Tests/\+Benchmarks}

\subparagraph*{Ubuntu 16.\+04/\+Debian 7}

Run the following to build the library test executables and the benchmarking apps \begin{DoxyVerb}make tests
make benchmarks
\end{DoxyVerb}


\subparagraph*{Cent\+OS 7/\+Redhat Enterprise Linux 7}

Run the following to build the library test executables and the benchmarking apps \begin{DoxyVerb}make rhel-test
make rhel-benchmarks
\end{DoxyVerb}


\subsection*{Use}

Please continue on to the \hyperlink{use_index}{Use} section of the documentation to see example uses of the library.

\hyperlink{index}{Go Home} 